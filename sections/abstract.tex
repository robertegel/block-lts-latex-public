% LTeX: enabled=false
%\section*{\IfLanguageName{english}{Abstract}{Zusammenfassung}}\label{sec:abstract}
\section*{Abstract}\label{sec:abstract}
% - Cconclusion kopieren
% - Drum herum beschreiben, was ist hms, warum wird was gemacht
% - was ist LTS, GTS
% - unterschied frozen flux / frozen block
% - 3/4 seite max
% LTeX: enabled=true

A \gls{lts} scheme for the \gls{hms}, a \acrlong{swe} solver developed at the Chair of Water Resources Management and Modelling of Hydrosystems, is introduced in this thesis.
A central goal of the development of \gls{hms} has been to increase computational performance.
Thus, the extension of \gls{hms} with an \gls{lts} scheme aligns well with this goal, as long as it 
% does not compromise 
is compatible with existing optimization methods within \gls{hms}.
This is why the proposed scheme operates on blocks of cells, instead of individual cells. 
It adopts the approach of reducing the amount of flux computations by \emph{freezing} flux vectors and reusing them according to a local stability criterion. 
This criterion is based on choosing the minimum local time step inside a block as the block's allowable time step, enabling the block's cells to be computed using parallel methods.
Thus, the proposed scheme is named \acrlong*{frozen-block-LTS}.
Reuse of previously computed fluxes is accomplished by adaptively scaling them according to the current global time step.
Initial problems with stability, as well as the synchronization of some adjacent blocks, were mitigated by a revision of the criterion for performing scalar flux computations and the introduction of neighbor propagation.
Except for one outlier case, \acrlong*{frozen-block-LTS} has shown good stability and exhibits comparable accuracy to the \gls{gts} scheme currently in use.
In an analytical benchmark, \acrlong*{frozen-block-LTS} was able to reduce simulation runtime by more than \SI{30}{\percent}, while in a practical rainfall-runoff scenario, runtime reductions greater than \SI{20}{\percent} were achieved.


%https://wordvice.com/which-tense-should-be-used-in-abstracts-past-or-present/