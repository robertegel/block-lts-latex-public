\section{Cross-Platform Portability}\label{sec:porting-background}

As \gls{hms} has been developed to run on Linux natively, ports to other operating systems were previously available as Docker containers only.
This can lead to reduced performance due to the need of running the container on a Linux kernel, which is virtualized on Windows and macOS machines.
% This can lead to reduced performance due to the container overhead.
% Docker uses the Linux kernel from "Windows subsystem for Linux", so it's not like a virtual machine, but rather almost native.
% "The newest version of WSL uses a subset of Hyper-V architecture to enable its virtualization." from https://learn.microsoft.com/en-us/windows/wsl/faq

Linux is the de facto standard operating system of scientific computing, especially \gls{hpc} clusters heavily rely on Linux operating systems \autocite{linux2017}.
Despite constantly gaining market share in the world of desktop computers, Linux has not yet found its way to average users \autocite{statcounter2023}. To increase compatibility with more student's personal devices, \gls{hms} was ported to macOS.

Since Darwin, macOS's underlying core OS, is POSIX certified \autocite{lucy2007,opengroup2024}, porting Open-Source software written for Linux (mostly POSIX compliant) is relatively simple.
This holds true, especially in comparison to Windows, which is mostly incompatible with many practices in UNIX-like or UNIX systems like Linux and macOS.

\gls{hms} is written in C++ (hence the name), a language widely used across numerous operating systems.
Although C++ is heavily standardized, compiler behavior can vary intensely. Additionally, operating systems offer optimized system libraries or \glspl{sdk}, which may differ slightly as well. These are usually utilized by higher level libraries like \texttt{Eigen} or the program itself.
To eliminate the risk of diverging compiler behavior, \gls{gcc} was chosen over Apple \texttt{clang}, the default C/C++ compiler for macOS.

There are differences between \gls{arm} and the \gls{x86}, which, despite having been minimized by \gls{sdk} developers, should still be addressed when writing portable code \autocite{apple-archdiff,apple-arm64code}.
This applies to software that is highly optimized for \gls{x86}, e.g., because it:
``\begin{enumerate*}[label=(\roman*)]
	\item Interacts with third-party libraries you don't own.
	\item Interacts with the kernel or hardware.
	\item Relies on specific [\gls{gpu}] behaviors.
	\item Contains assembly instructions.
	\item Manages threads or optimizes your app's multithreaded behavior.
	\item Contains hardware-specific assumptions or performance optimizations.
\end{enumerate*}'' \autocite{apple-porting}.
Only the first point applies to \gls{hms}, since it uses \gls{omp} for managing parallel execution and relies on \texttt{Eigen} for its performance optimized linear algebra operations.
% , \note{both of which have already been successfully ported to \gls{arm} architecture (Can I simply say that or do I need to provide a source?).}
\gls{omp} is a compiler-level feature and readily available in \gls{gcc} for \gls{arm}, while \texttt{Eigen} similarly has mature support for \gls{arm} since version 3.4 \autocite{eigen2021}.
In conclusion, porting \gls{hms} from \gls{x86} to \gls{arm} is rather trivial because it utilizes high-level abstractions provided by well-maintained and widely used third-party frameworks instead of applying low-level optimizations by itself.


\subsection*{Specifics of Porting \gls{hms} to macOS}\label{sec:porting-steps}

Versions used in the context of this thesis:
\begin{itemize}
  \item macOS 14.5 for arm64/Apple Silicon
  \item \gls{gcc} 13.2.0
  \item MacOSX14.2 \gls{sdk} 
  \item Apple XCode Command Line Tools 15.1.0
  \item Open MPI 5.0.3
\end{itemize}

% LTeX: enabled=false
Steps undertaken to enable the execution of \gls{hms} under macOS:

\begin{itemize}
  \item \texttt{Eigen::Index} is a type alias for \texttt{long} under macOS (both \gls{gcc} and \texttt{clang}), not \texttt{long long} → use of \texttt{static\_cast} in vtk/xml related functions (conditional changes using compile time flags, therefore there are no type casts on Linux)
  \item add dependencies in cmake (\texttt{mpiHelpers} requires \texttt{io\_util})
  \item use of shared library linkage, because static linkage did not work in some cases, since \texttt{gcc} does not install its own linker on macOS, but uses the one provided with XCode CLI Tools instead
        % \begin{itemize}
        %   \item \texttt{hms}
        %   \item \texttt{eigen-fmt}
        %   \item \texttt{vtk}
        % \end{itemize}
  \item extend cache info functionality to macOS commands
  \item adapt existing bash scripts
        \begin{itemize}
          \item use Unix Makefiles for Mac
          \item adapt \texttt{get\_n\_threads\_build()}
          \item set \texttt{CXX=g++-13} and \texttt{CC=gcc-13} flags (otherwise \gls{gcc}-wrappers around \texttt{clang} is used) 
          \item set \texttt{-ld\_classic} flag (for Apple \texttt{clang} version $\geq$ 15)
          \item add \texttt{install\_mac.sh} for easy setup including installation of dependencies
        \end{itemize}
\end{itemize}