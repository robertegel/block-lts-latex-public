% ==============
% math
% ==============
%index typesetting
\newcommand{\tn}[1]{\textnormal{#1}}
%% typesetting indexes upright by default ( would have to replace every instance of _\tn{} )
%\def\subinrm#1{\sb{\textnormal{#1}}}
%{\catcode`\_=13 \global\let_=\subinrm}
%\mathcode`_="8000
%\def\upsubscripts{\catcode`\_=12 } \def\normalsubscripts{\catcode`\_=8 }
%% the toggle for upright subscripts
%\upsubscripts
%% the toggle for italic subscripts
%%\normalsubscripts
%matrix typesetting
%with arrows
\newcommand{\matr}[1]{\vec{#1}} % with small arrow
%\newcommand{\matr}[1]{\overrightarrow{#1}} % bigger arrow that can span multiple characters
%without arrows
%\newcommand{\matr}[1]{\mathbf{#1}} % undergraduate algebra version
%\newcommand{\matr}[1]{#1}          % pure math version
%\newcommand{\matr}[1]{\bm{#1}}     % ISO complying version, needs bm package
% differentials
% \newcommand{\dif}		{\mathop{}\!\mathrm{d}}
% \newcommand{\laplace}	{\mathop{}\!\Delta}
% \newcommand{\diver}		{\mathop{}\!\mathrm{div}\mathop{}\!}
% \newcommand{\grad}		{\mathop{}\!\mathrm{grad}\mathop{}\!}
% \newcommand{\rot}		{\mathop{}\!\mathrm{rot}\mathop{}\!}
% \newcommand{\pfrac}[2]	{\frac{\partial #1}		{\partial #2}}
% \newcommand{\ppfrac}[2]	{\frac{\partial^2 #1}	{\partial #2^2}}
% \newcommand{\diffrac}[2]{\frac{\dif #1}			{\dif #2}}
%other symbols, quantities, and expressions
\newcommand{\Rey}{\mathit{R\!e}} % \! suppresses spacing between characters
\newcommand{\Fr}{\mathit{F\!r}}
\newcommand{\Eu}{\mathit{E\!u}}
\newcommand{\Ma}{\mathit{M\kern-3pt a}}
\newcommand{\We}{\mathit{W\kern-4pt e}}
\newcommand{\Sr}{\mathit{S\!r}}
\newcommand{\NN}{\textnormal{NN}}
\newcommand{\const}{\mathrm{const.}}
% \newcommand{\Cr}{\ensuremath{\mathrm{Cr}}}
\newcommand{\Cr}{\ensuremath{\textit{Cr}}}

% ==============
% tables
% ==============
% the command "multicolumn" is way too long
\newcommand{\mcol}[3]{\multicolumn{#1}{#2}{#3}}
% to overwrite dcolumns in header row
\newcommand{\hcel}[1]{\multicolumn{1}{c}{#1}}
% empty cells
\newcommand{\emptycells}[1]{\multicolumn{#1}{l}{}}
\newcommand\mc[1]{\multicolumn{1}{c}{#1}}
% horizontally centered columns
%\newcolumntype{C}[1]{>{\centering\arraybackslash}X}
\newcolumntype{M}[1]{>{\centering\arraybackslash}m{#1}}
\newcolumntype{P}[1]{>{\centering\arraybackslash}p{#1}}
% fixed width left, center, and right columns
\newcolumntype{L}[1]{>{\raggedright\let\newline\\\arraybackslash\hspace{0pt}}p{#1}}
\newcolumntype{C}[1]{>{\centering\let\newline\\\arraybackslash\hspace{0pt}}p{#1}}
\newcolumntype{R}[1]{>{\raggedleft\let\newline\\\arraybackslash\hspace{0pt}}p{#1}}
% variable width for tabularx
\newcolumntype{x}{>{\raggedright\arraybackslash}X}
\newcolumntype{z}{>{$}l<{$}} % math-mode version of "l" column type
\newcolumntype{q}{>{$}c<{$}} % math-mode version of "c" column type
\newcolumntype{y}{>{$}r<{$}} % math-mode version of "r" column type
% unit column
\newcolumntype{u}{>{$\left[}l<{\right]$}}
% math-mode version of "R", "C", and "L" column type
\newcolumntype{Z}[1]{>{$}L{#1}<{$}}
\newcolumntype{Q}[1]{>{$}C{#1}<{$}}
\newcolumntype{Y}[1]{>{$}R{#1}<{$}}
% centering on decimal point. Because of icomma package, the printed glyph has to be set to \mathord\mathcomma
\newcolumntype{d}[1]{D{,}{\mathord\mathcomma}{#1}}
% width of right column with explanations
\newlength{\ecolwidth}
\setlength{\ecolwidth}{0.4\textwidth}
% width of left column when there are no explanations
\newlength{\lcolwidth}
\setlength{\lcolwidth}{0.4\textwidth}

% ==============
% automation
% ==============
% subdirectories for importing files
\newcommand{\subf}{}
% ==============
% style and layout
% ==============
\newcommand{\spacedhrule}{%
	\vspace{0.2\baselineskip}
	\hrule
	\vspace{0.2\baselineskip}
}
\newcommand{\fspacedhrule}{%
	\vspace{12pt}
	\hrule
	\vspace{12pt}
}
% paragraph title with linebreak
% the starred variant will never show up in the TOC, the non-starred one may, if tocnumdepth is high enough
\makeatletter
\def\lbpar{\@ifstar\@lbpar\@starredlbpar}
\def\@lbpar#1{%
	\paragraph{#1}\mbox{}\medskip
}
\def\@starredlbpar#1{%
	\paragraph*{#1}\mbox{}\medskip
}
\makeatother
% shortcuts
\newcommand{\hms}{\textup{\mdseries\ensuremath{\textsf{hms}\!^{+\!+}}}}
% \newcommand{\hms}{\mdseries{}\sffamily{}hms\ensuremath{\!^{+\!+}}\,}

\definecolor{semilightgray}{gray}{0.6}
% todo list
\newlist{todolist}{itemize}{5}
\newcommand{\cmark}{\ding{51}}%
\newcommand{\xmark}{\ding{55}}%
\newcommand{\canceled}{\rlap{$\square$}{\color{lightgray}\large\hspace{1pt}\xmark}}
\newcommand{\done}{\rlap{$\square$}{\raisebox{2pt}{\color{PineGreen}\large\hspace{1pt}\cmark}}\hspace{-2.5pt}}
\newcommand{\doing}{\hspace{-2.5pt}\rlap{$\square$}{\raisebox{2pt}{\color{Peach}\large\hspace{-4pt}\ding{244}}}\hspace{2.5pt}}
\newcommand{\itemtodo}[1]{\item {\color{semilightgray} #1}}
\newcommand{\itemcanceled}[1]{\item[\canceled] {\color{lightgray} \sout{#1}}}
\newcommand{\itemdoing}[1]{\item[\doing] {\color{Peach} #1}}
\newcommand{\itemdone}[1]{\item[\done] {\color{PineGreen} #1}}
\newcommand{\note}[1]{{\color{Peach} #1}}
% hyperref modifications
\newcommand{\algorithmautorefname}{Algorithm}
\addto\extrasenglish{%
  % \def\subsectionautorefname{Unterkapitel}%
  \def\sectionautorefname{Section}
	\def\subsectionautorefname{Section}
  \def\subsubsectionautorefname{Section}
}